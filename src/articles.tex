\documentclass[10pt,a4paper,uplatex]{jsarticle}
\usepackage{bm}
\usepackage{graphicx}
\usepackage[truedimen,left=25truemm,right=25truemm,top=25truemm,bottom=25truemm]{geometry}
\usepackage{jpdoc}

\def\title{定款}

\newcounter{NumOfMembers}\setcounter{NumOfMembers}{1}
\newcounter{DocumentsPerMember}\setcounter{DocumentsPerMember}{1}
\def\Name#1{\ifcase#1\or
株式会社科学計算総合研究所\or%甲
\else\fi}

\def\Address#1{\ifcase#1\or
東京都千代田区丸の内二丁目2番3号\or%甲
\else\fi}

\def\Representative#1{\ifcase#1\or
代表取締役 井原遊\or%甲
\else\fi}

\def\BlankSignature#1{\ifcase#1\or
0\or%甲
0\else0\fi}

\begin{document}
\newpage
{\centering \Large\bf \title  \vskip 0em}
\vskip 2em
\subsection{総則}
\article{商号}
当会社は、株式会社科学計算総合研究所と称する。英文では、Research Institute for Computational Science Co. Ltd. と表示し、RICOS Co. Ltd. と略称する。
\article{目的}
当会社は、次の事業を営むことを目的とする。
\begin{enumerate}
  \item[1] 計算科学、計算機科学及び応用力学の分野における研究並びに知的財産の創出
  \item[2] 計算資源の管理、貸与、買取及び販売
  \item[3] 計算機の設計、開発、販売、買取、保守及び輸出入
  \item[4] ソフトウェアの開発、販売、買取、保守及び輸出入
  \item[5] コンピュータシミュレーションを活用した解析の請負
  \item[6] その他適法な一切の事業
\end{enumerate}
\article{本店の所在地}
当会社は、本店を東京都千代田区に置く。
\article{公告の方法}
当会社の公告は、電子公告とする。ただし、事故その他やむを得ない事由によって電子公告による公告をすることができない場合は、官報に掲載してする。


\subsection{株式}
\article{発行可能株式総数}
当会社の発行する株式の総数は、5億5千万株とし、このうち5億2千万は普通株式、3千万株はA種優先株式とする。
\article{株式の譲渡制限}
当会社の株式を譲渡により取得するには、当会社の承認を要する。ただし、当会社の株主に譲渡する場合には、承認をしたものとみなす。
\article{株券の不発行}
当会社は、株券を発行しない。
\article{名義書換}
株式取得者が株主名簿記載事項を株主名簿に記載又は記録するには、当会社所定の書式による請求書に、その取得した株式の株主として株主名簿に記載又は記録された者又はその相続人その他の一般承継人及び株式取得者が署名又は記名押印し共同して請求しなければならない。
\article{質権の登録及び信託財産の表示}
当会社の株式について質権の登録又は信託財産の表示を請求するには、当会社所定の書式による請求書に当事者が署名又は記名押印し、共同して請求しなければならない。その登録又は表示の抹消についても同様とする。
\article{手数料}
前二条に定める請求をする場合には、当会社所定の手数料を支払わなければならない。
\article{株主の住所等の届出}
当会社の株主及び登録された質権者又はその法定代理人若しくは代表者は、当会社所定の書式により、その氏名、住所及び印鑑を当会社に届け出なければならない。届出事項に変更を生じたときも、その事項につき、同様とする。
\article{株主名簿の閉鎖及び基準日}
当会社は、営業年度末日の翌日から定時株主総会の終結の日まで株主名簿の記載の変更を停止する。前項の他、株主又は質権者として権利を行使すべき者を確定するため必要があるときは、あらかじめ公告して、一定期間株主名簿の記載の変更を停止し、又は基準日を定めることができる。
\article{株式の分割、併合及び株主割当て等}
当会社は、株式の分割又は併合を行うときは、全ての種類の株式につき同一割合でこれを行う。
\term 当会社は、株主に株式無償割当て又は新株予約権(新株予約権付社債に付されたもの含む。以下本条において同じ。)の無償割当てを行うときは、普通株主には普通株式又は普通株式を目的とする新株予約権の無償割当てを、A種優先株主にはA種優先株式又はA種優先株式を目的とする新株予約権の無償割当てを、それぞれ同時に同一割合で同一の条件にて行うものとする。
\term 当会社は、株主に募集株式の割当てを受ける権利又は募集新株予約権の割当てを受ける権利を与えるときは、普通株主には普通株式又は普通株式を目的とする新株予約権の割当てを受ける権利を、A種優先株主にはA種優先株式又はA種優先株式を目的とする新株予約権の割当てを受ける権利を、それぞれ同時に同一割合で同一の条件にて与える。

\subsubsection{A種優先株式}
\article{優先配当}
当会社は、剰余金の配当(中間配当を含む。以下単に「配当」という。)を行うときは、A種優先株式の保有者(以下「A種優先株主」という。)又はA種優先株式の登録株式質権者(A種優先株主とあわせて、以下「A種優先株主等」という。)に対し、普通株式の保有者(以下「普通株主」という。)及び普通株式の登録株式質権者(普通株主とあわせて、以下「普通株主等」という。)に先立ち、各事業年度ごとにA種優先株式1株につき\ref{残余財産の分配}に定めるA種優先分配基準額(\ref{残余財産の優先分配基準額}に基づきA種優先分配基準額が調整された場合にはその調整後の金額を意味する。)の2%に相当する剰余金(以下「A種優先配当額」という。)を配当する。但し、既に当該配当の基準日と同じ事業年度中に設けられた基準日によりA種優先株主等に対して配当を行ったときは、その額を控除した額とする。なお、当会社がさらに配当を行う場合には、A種優先株式及び普通株式に対し1株当たり同額の配当をする。
\label{優先配当}
\term ある事業年度においてA種優先株主等に対して行う1株当たりの配当の額がA種優先配当額に達しない場合、当該不足額は翌事業年度以降に累積しない。
\term 第1項に基づくA種優先配当額の計算上生じた1円未満の端数は切り捨てるものとする。
\term 当会社が会社分割を行う場合において、会社法第758条第八号ロ若しくは同法第760条第七号ロに規定する剰余金の配当をするとき、又は同法第763条第1項第十二号ロ若しくは同法第765条第1項第八号ロに規定する剰余金の配当をするときは、前各項の定めを適用しない。
\article{残余財産の分配}
当会社は、残余財産を分配するときは、A種優先株主等に対し、普通株主等に先立ち、A種優先株式1株につき、A種優先株式1株当たりの払込金額である金135000円の1倍に相当する金額(以下「A種優先分配基準額」という。)を支払う。
\label{残余財産の分配}
\term 前項による分配の後なお残余財産がある場合には、普通株主等及びA種優先株主等に対して分配を行う。この場合、当会社は、A種優先株主等に対しては、前項の分配基準額に加え、A種優先株式1株につき、普通株主等に対して普通株式1株につき分配する残余財産に\ref{普通株式と引換えにする取得請求権}に定めるA種取得比率を乗じた額と同額の残余財産を分配する。
\term A種優先分配基準額は、下記の定めに従い調整される。
\label{残余財産の優先分配基準額}
\begin{enumerate}
\item A種優先株式の分割、併合又は無償割当てが行われたときは、A種優先分配基準額は以下のとおり調整される。なお、「分割・併合・無償割当ての比率」とは、株式の分割、併合又は無償割当て後の発行済株式総数(自己株式を除く。)を株式の分割、併合又は無償割当て前の発行済株式総数(自己株式を除く。)で除した数を意味するものとし、以下同じとする。
\begin{displaymath}
調整後分配基準額 = 当該調整前の分配基準額 \times \frac{1}{分割・併合・無償割当ての比率}
\end{displaymath}
\item A種優先株主に割当てを受ける権利を与えて株式の発行又は処分(株式無償割当てを除く。)を行ったときは、A種優先分配基準額は以下のとおり調整される。なお、下記算式の「既発行A種優先株式数」からは、当該発行又は処分の時点における当会社が保有する自己株式(A種優先株式のみ)の数を除外するものとし、自己株式を処分する場合は下記算式の「新発行A種優先株式数」は「処分する自己株式(A種優先株式)の数」と読み替えるものとする。 
\begin{displaymath}
調整後分配基準額 = \frac{\shortstack{既発行\\A種優先株式数} \times \shortstack{当該調整前\\分配基準額} + \shortstack{新発行\\A種優先株式数} \times \shortstack{1株当たり\\払込金額} }{既発行A種優先株式数+新発行A種優先株式数}
\end{displaymath}
\item 第一号及び第二号における調整額の算定上発生した1円未満の端数は切り捨てるものとする。
\end{enumerate}
\article{金銭と引換えにする取得請求権}
A種優先株主は、当会社が、事業譲渡又は会社分割により、当会社の事業の重要部分を第三者に移転させた場合(以下「事業移転買収」という。)には、かかる移転の効力発生日を初日として、かかる移転の効力発生日又はかかる移転の全ての対価の交付の完了日のいずれか遅い日から1か月を経過するまでの期間(以下、本条において「取得請求期間」という。)に限り、保有するA種優先株式の全部又は一部を取得しその取得と引換えに本条の定めにより金銭を交付することを当会社に請求することができる。なお、「当会社の事業の重要部分」の移転とは、当会社の直近の監査済みの計算書類における総資産の50%超の資産の移転を伴う場合又は当会社の直近の監査済みの計算書類において当該移転した事業にかかる当会社の売上が総売上の50%を超える場合を意味する。
\term 前項の請求は、対象とする株式を特定した書面を当会社に交付することにより行うものとし、取得請求期間の満了時に効力が生じるものとする。
\term 本条によるA種優先株式1株の取得と引換えに交付される金銭は、以下の各号のいずれか高い方の金額とする。なお、下記における「A種優先分配基準額」は、その調整が行われた場合には調整後の金額を意味するものとする。
\begin{enumerate}
  \item A種優先分配基準額
  \item 取得請求期間の満了時において当会社を清算し、その場合の残余財産の総額が第1項に定める事業譲渡又は会社分割の対価の金額(対価が金銭以外の場合は当会社により合理的に算定され、A種優先株式の発行済株式総数の過半数を有するA種優先株主(複数名で当該割合以上の保有比率となる場合を含む。)の承認を得た対価の時価相当額とする。)であると仮定した場合に、本定款の定めに基づき普通株主及びA種優先株主がそれぞれ分配を受けられる金額に従って算出されるA種優先株式1株当たりの金額
\end{enumerate}
\term 本条による取得の請求があった場合、当会社は取得請求期間の満了時において請求の対象となったA種優先株式を取得するものとし、直ちに第3項に定める1株当たりの金額に対象となる株式数を乗じた金額をA種優先株主に支払うものとする。但し、A種優先株主に支払うべき金額が会社法において支払可能な金額(以下「法定財源」という。)を超える場合には、法定財源を第3項で定める1株当たりの交付される金銭の額で除した株式数(1株未満の端数は切り捨てる。)についてのみ本条に基づく取得請求権の効力が生じるものとし、その他の株式については取得請求権の行使の効力は生じないものとする。また、複数のA種優先株主が同時に本条に基づく取得請求権を行使し、かつ、上記但書の適用を受ける場合には、各A種優先株主について取得請求権の効力が発生するべき株式の数は、各A種優先株主が取得請求権を行使した株式の数に応じて按分するものとする(なお、按分にあたり生じる1株未満の端数は切り捨て本条に基づく取得の請求の対象とはしないものとする。)。
\term 前各項に定めるほか、当会社が会社法第156条から第165条まで(株主との合意による取得)の定めに基づき自己株式の有償での取得を行う場合には、A種優先株主は、普通株式に優先してA種優先株式を取得の対象とすることを請求できるものとする。
\article{普通株式と引換えにする取得請求権}
A種優先株主は、A種優先株主となった時点以降いつでも、保有するA種優先株式の全部又は一部につき、当会社がA種優先株式を取得するのと引換えに普通株式を交付することを当会社に請求することができる権利(以下「取得請求権」という。)を有する。その条件は以下のとおりとする。
\label{普通株式と引換えにする取得請求権}
\begin{enumerate}
  \item A種優先株式の取得と引換えに交付する普通株式数\\
  A種優先株式1株の取得と引換えに交付する当会社の普通株式の株式数(以下「A種取得比率」という。)は次のとおりとする。かかる取得請求権の行使により各A種優先株主に対して交付される普通株式の数につき1株未満の端数が発生した場合はこれを切り捨て、金銭による調整を行う。
  \begin{displaymath}
  A種取得比率 = \frac{A種優先株式の基準価額}{取得価額}
  \end{displaymath}
  \item 前号に定めるA種優先株式の基準価額及び同号に定める取得価額(以下「取得価額」という。)は、いずれも当初135000円とする。
\end{enumerate}



\article{取得価額等の調整}
前条に定めるA種優先株式の基準価額及び取得価額は、以下の定めにより調整される。
\label{取得価額等の調整}
\begin{enumerate}
  \item 株式等の発行又は処分に伴う調整\\
    A種優先株式発行後、下記イ又はロに掲げる事由により当会社の株式数に変更を生じる場合又は変更を生じる可能性がある場合は、取得価額を、下記に定める調整式に基づき調整する。調整額の算定上発生した1円未満の端数は切り捨てるものとする。\\
    なお、下記の調整式で使用する「既発行株式数」は、調整後の取得価額を適用する日の前日における、(1)当会社の発行済普通株式数(自己株式を除く。)と、(2)発行済潜在株式等(自己株式を除く。)の全てにつき取得原因が当該日において発生したとみなしたときに交付される普通株式数との合計数を意味するものとする(但し、当該調整の事由による普通株式又は潜在株式等の発行又は処分の効力が上記適用日の前日までに生じる場合、当該発行又は処分される普通株式及び当該発行又は処分される潜在株式等の目的たる普通株式の数は算入しない。)。\\
    「潜在株式等」とは、取得請求権付株式、取得条項付株式、新株予約権、新株予約権付社債、その他その保有者若しくは当会社の請求に基づき又は一定の事由の発生を条件として普通株式を取得し得る地位を伴う証券又は権利(A種優先株式を目的とする新株予約権のように、複数回の請求又は事由を通じて普通株式を取得し得るものを含む。)を意味し、「取得原因」とは、潜在株式等に基づき当会社が普通株式を交付する原因となる保有者若しくは当会社の請求又は一定の事由を意味し、「潜在株式等取得価額」とは、普通株式1株を取得するために当該潜在株式等の取得及び取得原因の発生を通じて負担すべき金額を意味する。\\
    当会社が自己の保有する普通株式又は潜在株式等を処分することにより調整が行われる場合においては、下記の調整式で使用する「新発行株式数」は「処分する株式数」と読み替えるものとする。\\
    当会社が潜在株式等を発行又は処分することにより調整が行われる場合においては、下記の調整式で使用する「新発行株式数」とは、発行又は処分される潜在株式等の目的たる普通株式の数を、「1株当たり払込金額」とは、潜在株式等取得価額を、それぞれ意味するものとする。\\
    下記イ又はロに定める普通株式又は潜在株式等の発行又は処分が、株主割当て又は無償割当て(株式無償割当てを除く。)により行われる場合は、前条に定めるA種優先株式の基準価額も、取得価額と同様に調整されるものとする。また、かかる発行又は処分が実質的に株主に対する割当ての目的で形式上株主割当て又は無償割当て以外の手続により行われる場合も、当会社の取締役会の決議(取締役会設置会社でない場合には取締役の決定)に基づきA種優先株式の基準価額も同様に調整されるものとする。
    \begin{displaymath}
    調整後取得価額 = \frac{\shortstack{既発行株式数} \times \shortstack{当該調整前取得価額} + \shortstack{新発行株式数} \times \shortstack{1株当たり払込金額} }{既発行株式数+新発行株式数}
    \end{displaymath}

    \begin{enumerate}
      \item 調整前の取得価額を下回る払込金額をもって普通株式を発行又は処分する場合。但し、次に掲げる場合のいずれかに該当する場合は、この限りでない。調整後の取得価額は、募集又は割当てのための基準日があるときはその日の翌日、それ以外のときは株式の発行又は処分の効力発生日(会社法第209条第1項第二号が適用される場合は、同号に定める期間の末日)の翌日以降にこれを適用する。
      \begin{enumerate}
        \itm 株式無償割当てによる場合
        \itm A種優先株式の取得請求権の行使その他潜在株式等の取得原因の発生による場合
        \itm 合併、株式交換、株式交付若しくは会社分割により普通株式を交付する場合
        \itm 会社法第194条の規定(単元未満株主による単元未満株式売渡請求)に基づく自己株式の売渡しによる場合
      \end{enumerate}
      \item 調整前の取得価額を下回る潜在株式等取得価額をもって普通株式を取得し得る潜在株式等を発行又は処分する場合(新株予約権無償割当てを含むが、株式無償割当てを除く。また潜在株式等の取得原因の全部又は一部の発生による場合を除く。)。調整後の取得価額は、募集又は割当てのための基準日がある場合はその日、それ以外のときは潜在株式等の発行又は処分の効力発生日(会社法第209条第1項第二号が適用される場合は、同号に定める期間の末日)に、全ての潜在株式等につき取得原因が発生したものとみなし、このみなされる日の翌日以降これを適用する。
    \end{enumerate}


  \item 株式等の発行又は処分に伴う調整の例外\\
    前号の調整は、次に掲げる場合のいずれかに該当する場合には行われない。
    \begin{enumerate}
      \item 次に掲げる条件のいずれかを満たす新株予約権を、当会社又は当会社の子会社の役員及び使用人に対して、ストックオプション目的で発行する場合(当該発行直後において、当会社の発行する全ての新株予約権(新株予約権付社債に付されたものを除く。)の目的たる株式数の合計数が発行済株式総数の15%を超えない場合に限る。)
        \begin{enumerate}
          \item 1株あたりの行使価額が新株予約権発行時における当会社の普通株式の時価を下回らない新株予約権
          \item 払込金額が適切なオプション価格以上である新株予約権
        \end{enumerate}
      \item A種優先株式の発行済株式総数の過半数を有するA種優先株主(複数名で当該割合以上の保有比率となる場合を含む。)が書面により調整しないことに同意した場合
    \end{enumerate}

    \item 株式の分割、併合又は無償割当てによる調整\\
    A種優先株式発行後、株式の分割、併合又は無償割当てを行う場合は、取得価額は以下の調整式に基づき調整される。調整後の取得価額は、株式分割、株式併合又は株式無償割当ての効力発生日(割当てのための基準日がある場合はその日)の翌日以降、適用されるものとする。調整額の算定上発生した1円未満の端数は切り捨てるものとする。また、この場合A種優先株式の基準価額も、取得価額と同様に調整されるものとする。
      \begin{displaymath}
      調整後取得価額 = 当該調整前取得価額 \times \frac{1}{分割・併合・無償割当ての比率}
      \end{displaymath}
    \item その他の調整\\
    上記に掲げた事由によるほか、次に該当する場合には、当会社は取締役会の決議(取締役会設置会社でない場合には株主総会の決議)に基づき、合理的な範囲において取得価額及びA種優先株式の基準価額の双方又はいずれかの調整を行うものとする。但し、かかる調整は、当該調整事由が生じる前のA種優先株式の経済的価値を損なわない範囲でのみ行われるものとする。
      \begin{enumerate}
        \item 合併、会社分割、株式移転又は株式交換のために取得価額の調整を必要とする場合。
        \item 潜在株式等の取得原因が発生する可能性のある期間が終了した場合。但し、潜在株式等の全部について取得原因が発生した場合を除く。
        \item 潜在株式等にかかる第一号ロに定める潜在株式等取得価額が修正される場合。
        \item 上記のほか、当会社の普通株式数に変更又は変更の可能性を生じる事由の発生によって取得価額の調整が必要であると取締役会(取締役会設置会社でない場合には取締役)が判断する場合。
      \end{enumerate}
\end{enumerate}
\article{普通株式と引換えにする取得}
当会社は、A種優先株式の発行以降、当会社の株式のいずれかの金融商品取引所への上場(以下「株式公開」という。)の申請を行うことが取締役会(取締役会設置会社でない場合には株主総会)で承認され、かつ株式公開に関する主幹事の金融商品取引業者から要請を受けた場合には、取締役会(取締役会設置会社でない場合には株主総会)の定める日をもって、発行済のA種優先株式の全部を取得し、引換えにA種優先株主に当会社の普通株式を交付することができる。かかる場合に交付すべき普通株式の内容、数その他の条件については、\ref{普通株式と引換えにする取得請求権}及び\ref{取得価額等の調整}の定めを準用する。なお、A種優先株主に交付される普通株式の数に1株に満たない端数が発生した場合の処理については、会社法第234条に従うものとする。
\article{議決権}
A種優先株主は、当会社株主総会及びA種優先株主を構成員とする種類株主総会(以下「A種種類株主総会」という。)において、A種優先株式1株につき1個の議決権を有する。
\article{みなし清算}
本条において、「買収」とは、当会社が以下のいずれかに該当することを意味する。
\begin{enumerate}
\item 当会社の発行済株式の議決権総数の50%超を特定の第三者(当会社の株主を含む。以下本条において同じ。)が自ら並びにその子会社及び関連会社により取得すること。なお、「子会社」及び「関連会社」とは、財務諸表等の用語、様式及び作成方法に関する規則(昭和38年大蔵省令第59号、その後の改正も含む。)第8条で定義される「子会社」及び「関連会社」を意味する。但し、当該第三者並びにその子会社及び関連会社が、合算で、当該取得前から当会社の発行済株式の議決権総数の50%超を有していた場合を除く。
\item 当会社が第三者と合併することにより、合併直前の当会社の総株主が合併後の会社に関して保有することとなる議決権総数が、合併後の会社の発行済株式の議決権総数の50%未満となること。
\item 当会社が第三者と株式交換を行うことにより、株式交換直前の当会社の総株主が株式交換後の完全親会社に関して保有することとなる議決権総数が、株式交換後の完全親会社の発行済株式の議決権総数の50%未満となること。
\item 当会社が第三者と株式移転を行うことにより、株式移転直前の当会社の総株主が株式移転後の完全親会社に関して保有することとなる議決権総数が、株式移転後の完全親会社の発行済株式の議決権総数の50%未満となること。
\item 当会社が第三者と株式交付を行うことにより、株式交付直前の当会社の総株主が株式交付後の株式交付親会社に関して保有することとなる議決権総数が、株式交付後の株式交付親会社の発行済株式の議決権総数の50%未満となること。
\item 事業移転買収
\end{enumerate}
\term 当会社について買収(事業移転買収を除く。以下本項において同じ。)が行われる場合には、その買収の対価については、買収に応じた株主の間で以下の定めに基づき分配を行うものとする。
\begin{enumerate}
\item 買収の対価が現金の場合、買収の対価の合計額を残余財産とし、買収に応じた株主のみが当会社の株主である前提で当会社を清算したと仮定した場合に、本定款の定めに基づき普通株主及びA種優先株主がそれぞれ分配を受けられる金額に基づいて、各株主が分配を受けられる金額を算出し、その金額と同額の現金を買収の対価の分配として各株主の間で分配する。
\item 買収の対価が現金以外の場合、買収の対価について、A種優先株式の発行済株式総数の過半数を有するA種優先株主(複数名で当該割合以上の保有比率となる場合を含む。)が合理的に当該対価の評価額を算定し、買収の対価の合計額を残余財産とし、買収に応じた株主のみが当会社の株主である前提で当会社を清算したと仮定した場合に、本定款の定めに基づき普通株主及びA種優先株主がそれぞれ分配を受けられる金額に基づいて、各株主が分配を受けられる金額を算出し、その金額と同額の対価を買収の対価の分配として各株主の間で分配する。
\end{enumerate}
\term 当会社について会社分割の方法による事業移転買収が行われる場合において、会社法第758条第八号ロ若しくは同法第760条第七号ロに規定する剰余金の配当をするとき、又は同法第763条第1項第十二号ロ若しくは同法第765条第1項第八号ロに規定する剰余金の配当をするときには、当該配当される承継会社又は新設会社の株式について、A種優先株式の発行済株式総数の過半数を有するA種優先株主(複数名で当該割合以上の保有比率となる場合を含む。)が合理的に評価額を算定し、配当される株式の価額の合計額を残余財産として当会社を清算したと仮定した場合に、\ref{残余財産の分配}の定めに基づき普通株主及びA種優先株主がそれぞれ分配を受けられる金額に基づいて、各株主が分配を受けられる金額を算出し、その金額と同額の株式を各株主に配当する。かかる配当に関しては、\ref{優先配当}の規定は適用しないものとする。

\subsection{株主総会}
\article{招集}
当会社の定時株主総会は、営業年度末日の翌日から3か月以内に招集し、臨時株主総会は、必要に応じて招集する。
\article{招集手続きの省略}
株主総会は、その総会において、議決権を行使することができるすべての株主の同意があるときは、招集手続きを経ずに開催することができる。
\article{議長}
株主総会の議長は、代表取締役がこれに当たる。
\term 代表取締役に事故があるときは、他の取締役が議長になる。
\term 取締役全員に事故があるときは、総会において出席株主のうちから議長を選出する。
\article{決議の方法}
株主総会の決議は、法令又は定款に別段の定めがある場合の他、出席した株主の議決権の過半数をもって決する。株主総会の特別決議は、総株主の議決権の3分の1以上を有する株主が出席して、その議決権の3分の2以上をもって決する。
\article{書面による決議}
株主総会の決議の目的である事項について、取締役又は株主から提案があった場合には、その事項につき議決権を行使することができるすべての株主が、書面によってその提案に同意したときは、その提案を可決する総会の決議があったものとみなす。
\article{種類株主総会}
種類株主総会の決議は、法令又は本定款に別段の定めがある場合を除き、出席した議決権を行使することができる当該種類株主の議決権の過半数をもって行う。
\label{種類株主総会}
\term 会社法第324条第2項の定めによる種類株主総会の決議は、議決権を行使することができる当該種類株主の議決権の3分の1以上を有する種類株主が出席し、その議決権の3分の2以上をもって行う。
\term 株主総会に関する規定のうち、招集権者、議長、議決権の代理行使及び議事録に関する規定は種類株主総会に準用する。
\term 当会社は、会社法第322条第3項但書その他法令に別段の定めがある場合を除き、会社法第322条第1項に定める種類株主総会の決議を要しない。
\term 当会社は、法令に別段の定めがある場合を除き、会社法第199条第4項、第200条第4項、第238条第4項、第239条第4項及び第795条第4項に規定する事項について、種類株主総会の決議を要しない。
\article{A種種類株主総会の決議を要する事項に関する定め}
下記の各事項のうち、会社法又は本定款において株主総会決議事項とされていない事項は取締役会決議事項(取締役会設置会社でない場合には株主総会決議事項)とし、\ref{種類株主総会}の定めにかかわらず、当会社が下記の各事項を行うためには、取締役会又は株主総会の決議に加えて、A種種類株主総会の決議を得るものとする。但し、当会社の取締役全員が承認した事項については、この限りではない。
\begin{enumerate}
\item 定款の変更
\item 当会社の買収(特定の第三者(その者の関係会社を含む)による発行済株式の議決権総数の50%を超える株式又は潜在株式の取得をいう。)、合併、株式交換、株式移転、株式交付、事業譲渡、事業譲受、会社分割その他企業再編若しくは第三者との資本提携の実施、第三者からのこれらに関する提案の受諾又は拒絶
\item 株式、潜在株式(新株予約権、新株予約権付社債、新株引受権、転換社債、新株引受権付社債、その他株式への転換、株式との交換その他株式の取得が可能となる証券又は権利を意味する。以下本条において同じ。)又は社債の発行又は処分。但し、潜在株式にかかる権利の行使又は取得条項に基づき発行又は処分する場合を除く。
\item 株式又は潜在株式の譲渡又は取得に対する承認又は買取人の指定
\item 株式分割、株式併合、単元株の設定、その他の株主の地位又は権利に影響を及ぼす一切の事項
\item 自己株式の取得(但し、第三号但書の場合を除く。)、株式消却、資本金若しくは準備金の増加若しくは減少、又はその他の資本の変更
\item 潜在株式の条件の決定又は変更
\item 取締役、監査役及び会計監査人の選任又は解任
\item 解散、又は破産手続開始、民事再生手続開始、会社更生手続開始、特別清算開始その他これらに類する手続の開始の申立て
\item \ref{取得価額等の調整}第三号に基づく取得価額及びA種優先株式の基準価額の調整
\item 株式上場に関する引受主幹事証券又は監査法人(公認会計士)の決定又は変更
\end{enumerate}
\term 前項の事項に関するA種種類株主総会の決議は、\ref{種類株主総会}第一項及び第二項の定めにかかわらず、法令に別段の定めがある場合を除き、A種種類株主総会において議決権を行使することができるA種優先株主の議決権の過半数を有するA種優先株主が出席し、その議決権の3分の2以上をもって行う。

\subsection{取締役、代表取締役及び監査役}
\article{監査役の設置}
当会社に監査役を設置する。
\article{取締役及び監査役の員数}
当会社の取締役は1名以上とし、監査役は1名以上とする。
\article{取締役及び監査役の選任}
当会社の取締役及び監査役は、株主総会において議決権を行使することができる株主の議決権の数の3分の1以上の議決権を有する株主が出席し、その議決権の過半数の決議によって選任する。取締役の選任については、累積投票によらないものとする。
\article{取締役の任期}
取締役の任期は、就任後1年内の最終の決算期に関する定時株主総会の終結の時までとする。補欠又は増員により選任された取締役は、他の取締役の任期の残存期間と同一とする。
\article{監査役の任期}
監査役の任期は、就任後4年内の最終の決算期に関する定時株主総会の終結の時までとする。任期の満了前に退任した監査役の補欠として選任された監査役の任期は、退任した監査役の任期が満了すべき時までとする。
\article{代表取締役及び役付取締役}
株主総会は、その決議により取締役の中から代表取締役を1名以上定める。
\term 株主総会は、その決議により取締役の中から取締役会長、専務取締役及び常務取締役を定めることができる。
\article{報酬及び退職慰労金}
取締役及び監査役の報酬及び退職慰労金はそれぞれ株主総会の決議をもって定める。
\article{責任限定契約}
当会社は、会社法第427条第1項の規定により、非業務執行取締役等との間で、同法第423条第1項の賠償責任を限定する契約を締結することができる。ただし、当該契約に基づく責任の限度額は、法令が規定する額とする。

\subsection{計算}
\article{事業年度}
当会社の事業年度は、毎年11月1日から翌年の10月31日までの年1期とする。
\article{利益配当}
利益配当金は、毎営業年度末日現在における株主名簿に記載された株主又は質権者に対して支払う。利益配当金が、その支払提供の日から満3年を経過しても受領されないときは、当会社はその支払義務を免れるものとする。


%CERTIFICATION FIELD%
\vspace{20pt}
上記は当会社の現行定款と相違ない。
\begin{flushleft} 
\today\\
\vspace{10pt}
\MakeSignatureField
\end{flushleft}
%CERTIFICATION FIELD%
\end{document}
