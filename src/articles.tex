\documentclass[10pt,a4paper,uplatex]{jsarticle}
\usepackage{bm}
\usepackage{graphicx}
\usepackage[truedimen,left=25truemm,right=25truemm,top=25truemm,bottom=25truemm]{geometry}
\usepackage{jpdoc}

\def\title{定款}

\newcounter{NumOfMembers}\setcounter{NumOfMembers}{1}
\newcounter{DocumentsPerMember}\setcounter{DocumentsPerMember}{1}
\def\Name#1{\ifcase#1\or
発起人 井原 良忠\or%甲
\else\fi}

\def\Address#1{\ifcase#1\or
兵庫県加古川市加古川町本町363番地 \or%甲
\else\fi}

\def\Representative#1{\ifcase#1\or
\or%甲
\else\fi}

\def\BlankSignature#1{\ifcase#1\or
0\or%甲
0\else0\fi}

\begin{document}
\newpage
{\centering \Large\bf \title  \vskip 0em}
\vskip 2em
\subsection{総則}
\article{商号}
当会社は、株式会社井原良忠事務所と称する。

\article{目的}
当会社は、次の事業を営むことを目的とする。
\begin{enumerate}
  \item[1] 植物の栽培及び研究
  \item[2] 植物による緑化製品の企画・開発および生産
  \item[3] 植物による緑化製品の生産技術者の養成機関の運営
  \item[4] デザイン設計者の養成機関の運営
  \item[5] 環境芸術の企画及び制作・著作
  \item[6] 公共緑地の調査・計画・設計・施工管理
  \item[7] 庭園の調査・計画・設計・施工管理
  \item[8] 環境計画の企画及び設計・施工管理
  \item[9] 建築デザインの企画設計および生産管理
  \item[10] エクステリア・インテリア商品の企画設計および生産管理
  \item[11] 国外への造園土木資材の輸出及び輸入
  \item[12] 国外への建築資材の輸出及び輸入
  \item[13] 国外への著作物の輸出及び輸入
  \item[14] 前各号に付帯関連する一切の業務
\end{enumerate}
\article{本店の所在地}
当会社は、本店を兵庫県加古川市に置く。
\article{公告の方法}
当会社の公告は、官報に掲載してする。

\subsection{株式}
\article{発行する株式の総数}
当会社の発行する株式の総数は、1200株とする。
\article{株式の譲渡制限}
当会社の株式を譲渡するには、取締役会の承認を受けなければならない。
\article{株券の発行}
当会社は、株式に係る株券を発行する。
\article{名義書換}
株式の取得により名義書換を請求するには、当会社所定の書式による請求書に記名押印し、これに次の書面を添えて提出しなければならない。
\begin{enumerate}
  \item 譲渡による株式の取得の場合には、株券
  \item 譲渡以外の事由による株式の取得の場合には、その取得を証す書面及び株券
\end{enumerate}
\article{質権の登録及び信託財産の表示}
当会社の株式について質権の登録又は信託財産の表示を請求するには、当会社所定の書式による請求書に当事者が署名又は記名押印し、共同して請求しなければならない。その登録又は表示の抹消についても同様とする。
\article{手数料}
前三条に定める請求をする場合には、当会社所定の手数料を支払わなければならない。
\article{株主の住所等の届出}
当会社の株主及び登録された質権者又はその法定代理人若しくは代表者は、当会社所定の書式により、その氏名、住所及び印鑑を当会社に届け出なければならない。届出事項に変更を生じたときも、その事項につき、同様とする。
\article{株主名簿の閉鎖及び基準日}
当会社は、営業年度末日の翌日から定時株主総会の終結の日まで株主名簿の記載の変更を停止する。前項の他、株主又は質権者として権利を行使すべき者を確定するため必要があるときは、あらかじめ公告して、一定期間株主名簿の記載の変更を停止し、又は基準日を定めることができる。

\subsection{株主総会}
\article{招集}
当会社の定時株主総会は、営業年度末日の翌日から3か月以内に招集し、臨時株主総会は、必要に応じて招集する。
\article{招集手続きの省略}
株主総会は、その総会において、議決権を行使することができるすべての株主の同意があるときは、招集手続きを経ずに開催することができる。
\article{議長}
株主総会の議長は、社長がこれに当たる。社長に事故があるときは、あらかじめ取締役会の定める順序により、他の取締役がこれに代わる。
\article{決議の方法}
株主総会の決議は、法令又は定款に別段の定めがある場合の他、出席した株主の議決権の過半数をもって決する。株主総会の特別決議は、総株主の議決権の3分の1以上を有する株主が出席して、その議決権の3分の2以上をもって決する。
\article{書面による決議}
株主総会の決議の目的である事項について、取締役又は株主から提案があった場合には、その事項につき議決権を行使することができるすべての株主が、書面によってその提案に同意したときは、その提案を可決する総会の決議があったものとみなす。

\subsection{取締役、取締役会、代表取締役及び監査役}
\article{取締役会の設置}
当会社に取締役会を設置する。
\article{監査役の設置}
当会社に監査役を置く。
\article{取締役及び監査役の員数}
当会社の取締役は3名とし、監査役は1名とする。
\article{取締役及び監査役の選任}
当会社の取締役及び監査役は、株主総会において議決権を行使することができる株主の議決権の数の3分の1以上の議決権を有する株主が出席し、その議決権の過半数の決議によって選任する。取締役の選任については、累積投票によらないものとする。
\article{取締役及び監査役の任期}
取締役及び監査役の任期は、就任後10年内の最終の決算期に関する定時株主総会の終結の時までとする。補欠又は増員により選任された取締役は、他の取締役の任期の残存期間と同一とする。任期の満了前に退任した監査役の補欠として選任された監査役の任期は、退任した監査役の任期が満了すべき時までとする。
\article{監査の範囲}
監査役の監査の範囲は、会計に関するものに限定する。 
\article{報酬及び退職慰労金}
取締役及び監査役の報酬及び退職慰労金はそれぞれ株主総会の決議をもって定める。


\subsection{計算}
\article{事業年度}
当会社の事業年度は、毎年4月1日から翌年の3月31日までの年1期とする。
\article{利益配当}
利益配当金は、毎営業年度末日現在における株主名簿に記載された株主又は質権者に対して支払う。利益配当金が、その支払提供の日から満3年を経過しても受領されないときは、当会社はその支払義務を免れるものとする。

\subsection{附則}
\article{設立に際して発行する株式}
当会社の設立に際して発行する株式の総数は、普通株式300株とし、その発行価額は、1株につき1万円とする。

\article{最初の営業年度}
当会社の最初の営業年度は、当会社成立の日から平成19年3月31日までとする。

\article{最初の取締役及び監査役の任期}
当会社の最初の取締役及び監査役の任期は、就任後1年内の最終の決算期に関する定時株主総会の終結の時までとする。

\article{発起人の氏名、住所及び引受株数}
発起人の氏名、住所及び発起人が引き受けた株式の数は、次のとおりである。
\begin{table}[h]
\begin{tabular}{|c|c|c|}\hline
氏名        &  住所                                  & 株式数             \\ \hline
井原 良忠  & 兵庫県加古川市加古川町本町363番地   & 普通株式300株   \\ \hline
\end{tabular}
\end{table}

%CERTIFICATION FIELD%
\vspace{20pt}
以上、株式会社井原良忠事務所の設立のため、この定款を作成し、発起人が次に記名押印する。
\begin{flushleft} 
\today\\
\vspace{10pt}
\MakeSignatureField
\end{flushleft}
%CERTIFICATION FIELD%
\end{document}
